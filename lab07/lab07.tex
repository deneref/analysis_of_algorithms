\documentclass[12pt]{report}
\usepackage[utf8]{inputenc}
\usepackage[russian]{babel}
%\usepackage[14pt]{extsizes}
\usepackage{listings}

% Для листинга кода:
\lstset{ %
language=go,                 % выбор языка для подсветки 
basicstyle=\small\sffamily, % размер и начертание шрифта для подсветки кода
numbers=left,               % где поставить нумерацию строк (слева\справа)
numberstyle=\tiny,           % размер шрифта для номеров строк
stepnumber=1,                   % размер шага между двумя номерами строк
numbersep=5pt,                % как далеко отстоят номера строк от подсвечиваемого кода
showspaces=false,            % показывать или нет пробелы специальными отступами
showstringspaces=false,      % показывать или нет пробелы в строках
showtabs=false,             % показывать или нет табуляцию в строках            
tabsize=2,                 % размер табуляции по умолчанию равен 2 пробелам
captionpos=t,              % позиция заголовка вверху [t] или внизу [b] 
breaklines=true,           % автоматически переносить строки (да\нет)
breakatwhitespace=false, % переносить строки только если есть пробел
escapeinside={\#*}{*)}   % если нужно добавить комментарии в коде
}

% Для измененных титулов глав:
\usepackage{titlesec, blindtext, color} % подключаем нужные пакеты
\definecolor{gray75}{gray}{0.75} % определяем цвет
\newcommand{\hsp}{\hspace{20pt}} % длина линии в 20pt
% titleformat определяет стиль
\titleformat{\chapter}[hang]{\Huge\bfseries}{\thechapter\hsp\textcolor{gray75}{|}\hsp}{0pt}{\Huge\bfseries}

%отступы по краям
\usepackage{geometry}
\geometry{verbose, a4paper,tmargin=2cm, bmargin=2cm, rmargin=1.5cm, lmargin = 3cm}
% межстрочный интервал
\usepackage{setspace}
\onehalfspacing
\usepackage{float}
% plot
\usepackage{pgfplots}
\usepackage{filecontents}
\usepackage{amsmath}
\usepackage{tikz,pgfplots}
\usetikzlibrary{datavisualization}
\usetikzlibrary{datavisualization.formats.functions}

\usepackage{graphicx}
\graphicspath{{src/}}
\DeclareGraphicsExtensions{.pdf,.png,.jpg}

\usepackage{geometry}
\geometry{verbose, a4paper,tmargin=2cm, bmargin=2cm, rmargin=1.5cm, lmargin = 3cm}
\usepackage{indentfirst}
\setlength{\parindent}{1.4cm}

\usepackage{titlesec}
\titlespacing{\chapter}{0pt}{12pt plus 4pt minus 2pt}{0pt}

\begin{filecontents}{timeAnts.dat}
	2 6.9806
	3 14.96
	4 22.9391
	5 35.9015
	6 52.8592
	7 71.8331
	8 93.7493
	9 117.7173
	10 147.609
\end{filecontents}
\begin{filecontents}{timeBrute.dat}
	2 0 
	3 0.9
	4 1
	5 1.4
	6 1.4
	7 2
	8 11.0021
	9 157.5722
	10 1523.9153
\end{filecontents}

\begin{document}
%\def\chaptername{} % убирает "Глава"
\begin{titlepage}
	\centering
	{\scshape\LARGE МГТУ им. Баумана \par}
	\vspace{3cm}
	{\scshape\Large Лабораторная работа №7\par}
	\vspace{0.5cm}	
	{\scshape\Large По курсу: "Анализ алгоритмов"\par}
	\vspace{1.5cm}
	{\huge\bfseries Поиск подстроки в строке\par}
	\vspace{2cm}
	\Large Работу выполнил: Мокеев Даниил, ИУ7-54\par
	\vspace{0.5cm}
	\Large Преподаватели:  Волкова Л.Л., Строганов Ю.В.\par

	\vfill
	\large \textit {Москва, 2019} \par
\end{titlepage}

\tableofcontents

\newpage
\chapter*{Введение}
\addcontentsline{toc}{chapter}{Введение}

Муравьиный алгоритм — один из эффективных полиномиальных алгоритмов для нахождения приближённых решений задачи коммивояжёра, а также решения аналогичных задач поиска маршрутов на графах.

Целью данной лабораторной работы является изучение муравьиных алгоритмов и приобретение навыков параметризации методов на примере муравьиного алгоритма, примененного к задаче коммивояжера.

Задачи данной лабораторной работы:
\begin{itemize}
	\item рассмотренть муравьиный алгоритм и алгоритм полного перебора в задаче коммивояжера;
	\item реализовать эти алгоритмы;
	\item сравнить время работы этих алгоритмов.
\end{itemize}


\chapter{Аналитическая часть}
В данной части будут рассмотрены существующие на данный момент алгоритмические решения проблемы поиска подстроки в строке. 

\section{Общие сведения об алгоритмах поиска подстроки}
\par
Поиск подстроки в строке — одна из простейших задач поиска информации. Сферы применения алгоритмов поиска включают в себя:
\begin{itemize}
	\item Текстовые редакторы;
	\item СУБД;
	\item компиляторы;
	\item программы определения проверки плагиата;
	\item поисковые системы;
	\item биоинформатика.
\end{itemize}
На сегодняшний день существует огромное разнообразие алгоритмов поиска подстроки. Программисту приходится выбирать подходящий в зависимости от таких факторов: длина строки, в которой происходит поиск, необходимость оптимизации, размер алфавита, возможность проиндексировать текст, требуется ли одновременный поиск нескольких строк.  В данной лабораторной работе будут рассмотремы два алгоритма сравнения с образцом, алгоритм Кнута-Морриса-Пратта и алгоритм Бойера-Мура.
\subsection{Стандартный алгоритм}
Стандартный алгоритм начинает со сравнения первого символа текста с первым символом подстроки. Если они совпадают, то происходит переход ко второму символу текста и подстроки. При совпадении сравниваются следующие символы. Так продолжается до тех пор, пока не окажется, что подстрока целиком совпала с отрезком текста, или пока не встретятся несовпадающие символы. В первом случае задача решена, во втором мы сдвигаем указатель текущего положения в тексте на один символ и заново начинаем сравнение с подстрокой\cite{2}.

\subsection{Алгоритм Бойера-Мура}
Алгоритм Бойера-Мура осуществляет сравнение с образцом справа налево, а не слева направо. Исследуя искомый образец, можно осуществлять более эффективные прыжки в тексте при обнаружении несовпадения. В этом алгоритме кроме таблицы суффиксов применяется таблица стоп-символов. Она заполняется для каждого сивола в алфавите. Для каждого встречающегося в подстроке символа таблица заполняется по принципу максимальной позиции символа в строке, за исключением последнего символа. При определении сдвига при очередном несовпадении строк, выбирается максимальное значение из таблицы суффиксов и стоп-символов\cite{2}.

\begin{center}
Таблица 2. Пошаговая работа алгоритма Бойера-Мура.\\
\begin{tabular}{| c | c | c | c | c | c | c | c | c | c | }
	\hline
	a&b&a&b&a&c&a&b&a&a \\
	\hline
	\hline
	a&b&a&\textcolor{red}{a}&&&&&&\\
	\hline
	&&a&b&a&\textcolor{red}{a}&&&&\\
	\hline
	&&&&&&\textcolor{green}{a}&b&a&a\\
	\hline
\end{tabular}
\end{center}


\subsection{Алгоритм Кнута-Морриса-Пратта}
Алгоритм Кнута-Морриса-Пратта основан на принципе конечного автомата, однако он использует более простой метод обработки неподходящих символов. В этом алгоритме состояния помечаются символами, совпадение с которыми должно в данный момент произойти. Из каждого состояния имеется два перехода: один соответствует успешному сравнению, другой - несовпадению. Успешное сравнение переводит нас в следующий узел автомата, а в случае несовпадения мы попадаем в предыдущий узел, отвечающий образцу. 
В программной реализации этого алгоритма применяется массив сдвигов, который создается для каждой подстроки, которая ищется в тексте. Для каждого символа из подстроки рассчитывается значение, равное максимальной длине совпадающего префикса и суффикса отсительно конкретного элемента подстроки. Создание этого массива позволяет при несовпадении строки сдвигать ее на расстояние, большее, чем 1 (в отличие от стандартного алгоритма).

\begin{center}
	Таблица 1. Пошаговая работа алгоритма Кнута-Морриса-Пратта.\\
	
	\begin{tabular}{| c | c | c | c | c | c | c | c | c | c | }
		\hline
		a&b&a&b&a&c&a&b&a&a \\
		\hline
		\hline
		a&b&a&\textcolor{red}{a}&&&&&&\\
		\hline
		&&a&b&a&\textcolor{red}{a}&&&&\\
		\hline
		&&&&a&\textcolor{red}{b}&a&a&&\\
		\hline
		&&&&&\textcolor{red}{a}&b&a&a&\\
		\hline
		&&&&&&a&b&a&\textcolor{green}{a}\\
		\hline
	\end{tabular}
\end{center}

\section*{Вывод}
\addcontentsline{toc}{section}{Введение}
В данном разделе были рассмотрены алгоритмы для решения задачи поиска подстроки в строке. 


\chapter{Конструкторская часть}
В данном разделе будут рассмотрены основные требования к программе и схемы алгоритмов.

\section{Требования к программе}
\textbf{Требования к вводу:}
\begin{itemize}
	\item Подаются не пустые подстрока и строка;
	\item длина подстроки меньше, чем длина строки.
\end{itemize}

\textbf{Требования к программе:}
\begin{itemize}
	\item Программа должна возвращать первое индекс первого вхождения подстроки.
\end{itemize}
.  
\newline  
\textbf{Входные данные} - на вход подается строка и подстрока;
\newline
\textbf{Выходные данные} - программа возвращает индекс первого вхождения подстроки в строку, если вхождения не было возвращается -1.

\section{Схемы алгоритмов}
В данном разделе будут приведены схемы алгоритмов для решения задачи коммивояжора:
полный пребор(Рис.\ref{fig:f_p}) и муравьиный (Рис. \ref{fig:ant})\\
\newpage

\section*{Вывод}
\addcontentsline{toc}{section}{Вывод}
В данном разделе были рассмотрены требования к программе и схемы алгоритмов.


\chapter{Технологическая часть}

\section{Выбор ЯП}
В качестве языка программирования был выбран golang.
Время работы алгоритмов было замерено с помощью time. 
\section{Листинг кода алгоритмов}
В данном разделе будут приведены листинги кода полного перебора всех решений (Листинг \ref{brute}) и реализации муравьиного алгоритма (Листинг \ref{ants})
\begin{lstlisting}[label=brute,caption = Перебор всех возможных вариантов, language = go]

func brute(file_name string) []int{
	weight := get_weights(file_name)
	path := make([]int, 0)
	res := make([]int, len(weight))

	for k:=0; k<len(weight);k++{
		ways := make([][]int, 0)
		_ = go_route(k, weight, path, &ways)
		sum := 1000
		curr := 0
		ind := 0
		for i:=0; i<len(ways);i++{
			curr = 0
			for j:=0;j<len(ways[i])-1;j++{
				curr+=weight[ways[i][j]][ways[i][j+1]]
			}
			if curr < sum{
				sum = curr
			}
		}
		res[k] = sum
	}
	return res
}

func contains(s []int, e int) bool {
	for _, a := range s {
		if a == e {
			return true
		}
	}
	return false
}


func go_route(pos int, weight [][]int, path[]int, routes *[][]int)[]int{
	path = append(path, pos)
	if len(path) < len(weight){
		for i:=0; i < len(weight); i++{
			if !(contains(path, i)){
				_ = go_route(i, weight,path, routes)
			}
		}
		}else{
			*routes = append(*routes, path)
	}
	return path
}
\end{lstlisting}

\begin{lstlisting}[label=ants,caption = Муравьиный алгоритм, language = go]

func start (env *env, days int) []int{
	shortest_dist := make([]int, len(env.weight))
	for n := 0; n < days; n++{
		for i:= 0; i< len(env.weight); i++{
			ant := env.new_ant(i)
			ant.ant_go()
			
			cur_dist := ant.get_distance()
			if (shortest_dist[i] == 0) || (cur_dist < shortest_dist[i]){
				shortest_dist[i] = cur_dist
			}
		}
	}
	return shortest_dist
}

func (ant *ant) ant_go(){
	for{
		prob := ant.count_probapility()
		choosen_path := choose_path(prob)
		if choosen_path == -1{
			break}
		ant.go_path(choosen_path)
		ant.renew_pheromon()
	}
}

func (ant *ant)count_probapility() []float64{
	res := make([]float64, 0);
	var d float64;
	var sum float64;
	for i, lenght := range ant.visited[ant.position]{
		if lenght != 0{
			d = math.Pow((float64(1)/float64(lenght)), ant.env.alpha) * math.Pow(ant.env.pheromon[ant.position][i], ant.env.betta)
			res = append(res, d)
			sum += d
		} else{
			res = append(res, 0)
		}
		}
		for _, lenght := range res{
			lenght = lenght / sum
	}
	return res
}

func choose_path(probab []float64) int{
	var sum float64
	for _, j := range probab{
		sum += j
	}
	r := rand.New(rand.NewSource(time.Now().UnixNano()))
	random_fl := r.Float64() * sum
	sum = 0
	for i , j := range probab{
		if random_fl > sum && random_fl<sum+j{
			return i
		} else{
		sum+=j
		}
	}
	return -1
}
\end{lstlisting}
\section*{Вывод}
\addcontentsline{toc}{section}{Вывод}
В данном разделе были рассмотрены основные сведения о модулях программы и листинг кода алгоритмов.

\chapter{Исследовательская часть}
В даннном разделе будет проведен сравнительный временной анализ алгоритмов и рассмотрена параметризация муравьиного алгоритма. Замеры времени были произведены на: Intel Core i5-6200U.
\chapter{Исследование зависимости времени работы алгоритмов от размера графа}
В данном разделе будет приведены результаты сравнения времени работы реализованных алгоритмов в зависимоти от размера матрицы смежности (Рис. \ref{plot:time}). Время измерено в миллисекундах.
\begin{figure}[!h]
	\begin{tikzpicture}[thick, scale=1.4]
	\begin{axis}[
	axis lines = left,
	xlabel = Количество вершин,
	ylabel = Время(миллисекунды),
	legend pos=north west,
	ymajorgrids=true
	]
	\addplot[color=red] table[x index=0, y index=1] {timeBrute.dat}; 
	\addplot[color=green] table[x index=0, y index=1] {timeAnts.dat};
	
	\addlegendentry{Полный перебор}
	\addlegendentry{Муравьиный алгоритм}
	\end{axis}
	\end{tikzpicture}
	\caption{Сравнение параллельного и обычного алгоритмов} \label{plot:time}
\end{figure}
\par

\chapter{Выводы исследовательского раздела}
Была исследована зависимоть времени работы реализованных алгоритмов от размера матрицы смежности графа. По результатам эксперимента на малых размерах графа полный перебор значительно выигрывает муравьиных алгоритм в скорости, однако на размера графа больше 8 сложность полного перебора растет очень быстро, а так как муравьиный алгоритм обладает полиноминальной сложностью, он работает быстрее перебора.

\chapter*{Заключение}
\addcontentsline{toc}{chapter}{Заключение}
В ходе лабораторной работы я изучила возможности применения и реализовала алгоритм полного перебора и муравьиный алгоритм. 

Временной анализ показал, что неэффективно использовать полный перебор на графе размера больше 8.

\addcontentsline{toc}{chapter}{Список литературы}
\begin{thebibliography}{3}
	\bibitem{1} Окулов С. М. Алгоритмы обработки строк. — М.: Бином, 2013. — 255 с.
	\bibitem{2} Дж. Макконнелл. Анализ лгоритмов. Активный обучающий подход
	\bibitem{UnitTests}
	Основные сведения о модульных тестах [Электронный ресурс], - режим доступа: https://docs.microsoft.com/ru-ru/visualstudio/test/unit-test-basics?view=vs-2019
\end{thebibliography}


\end{document}

